\documentclass[main.tex]{subfiles}
\begin{document}
\chapter{Particulars}\label{ch:howitworks}

\section{subfile package}
The subfile package\footnote{\url{http://ftp.math.purdue.edu/mirrors/ctan.org/macros/latex/contrib/subfiles/subfiles.pdf}} is a much more efficient way to bring in the sub parts of a large latex document.
Each part of the complete document brought in via \textbf{\textbackslash subfile\{file.tex\}} and \textbf{file.tex}:

\noindent Has a document class with the following structure:\newline\indent\textbackslash documentclass[\textbf{mainfile.tex}]\{\textbf{subfiles}\}



\displayIf[false]{
    \begin{figure}
        Fig 1
        \caption{Fig 1 does not gets compiled}
    \end{figure}
}

\displayIf{
    \begin{figure}
        Fig 2
        \caption{Fig 2 gets compiled}
    \end{figure}
}  

\displayIf{
    \begin{table}
        Table 1
        \caption{Table 1 gets compiled}
    \end{table}
}

\displayIf[false]{
    \begin{figure}
        Fig 2
        \caption{Table 2 does not gets compiled}
    \end{figure}
}  
\end{document}